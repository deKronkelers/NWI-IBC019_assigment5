\documentclass[12pt]{article}

\usepackage{enumerate}
\usepackage{calc}

\title{Assignment 5}

\author{Hendrik Werner s4549775}

\begin{document}
\maketitle

\section*{7.7}
	\begin{itemize}
		\item Initial state

		\begin{tabular}{|p{\linewidth}|}
			\hline
			512K\\
			\hline
		\end{tabular}

		Total internal fragmentation: $0$

		\item Allocating A (100K)

		\begin{tabular}{|p{\linewidth / 4}|p{\linewidth / 4}|p{\linewidth / 2}|}
			\hline
			A (100K / 128K) & 128K & 256K\\
			\hline
		\end{tabular}

		Total internal fragmentation: $28K$

		\item Allocating B (40K)

		\begin{tabular}{|p{\linewidth / 4}|p{\linewidth / 8}|p{\linewidth / 8}|p{\linewidth / 2}|}
			\hline
			A (100/128K) & B (40/64K) & 64K & 256K\\
			\hline
		\end{tabular}

		Total internal fragmentation: $28K + 24K = 52K$

		\item Allocating C (190K)

		\begin{tabular}{|p{\linewidth / 4}|p{\linewidth / 8}|p{\linewidth / 8}|p{\linewidth / 2}|}
			\hline
			A (100/128K) & B (40/64K) & 64K & C (190/256K)\\
			\hline
		\end{tabular}

		Total internal fragmentation: $28K + 24K + 66K = 118K$

		\item Freeing A

		\begin{tabular}{|p{\linewidth / 4}|p{\linewidth / 8}|p{\linewidth / 8}|p{\linewidth / 2}|}
			\hline
			128K & B (40/64K) & 64K & C (190/256K)\\
			\hline
		\end{tabular}

		Total internal fragmentation: $24K + 66K = 90K$

		\item Allocating D (60K)

		\begin{tabular}{|p{\linewidth / 4}|p{\linewidth / 8}|p{\linewidth / 8}|p{\linewidth / 2}|}
			\hline
			128K & B (40/64K) & D (60/64K) & C (190/256K)\\
			\hline
		\end{tabular}

		Total internal fragmentation: $24K + 4K + 66K = 94K$

		\item Freeing B

		\begin{tabular}{|p{\linewidth / 4}|p{\linewidth / 8}|p{\linewidth / 8}|p{\linewidth / 2}|}
			\hline
			128K & 64K & D (60/64K) & C (190/256K)\\
			\hline
		\end{tabular}

		Total internal fragmentation: $4K + 66K = 70K$

		\item Freeing D

		\begin{tabular}{|p{\linewidth / 2}|p{\linewidth / 2}|}
			\hline
			256K & C (190/256K)\\
			\hline
		\end{tabular}

		Total internal fragmentation: $66K$

		\item Freeing D

		\begin{tabular}{|p{\linewidth}|}
			\hline
			512K\\
			\hline
		\end{tabular}

		Total internal fragmentation: $0$

	\end{itemize}

\section*{7.10}
\begin{enumerate}[a]
	\item %a
	Yes, the sequence could be used to implement a buddy allocation system. You would start at fibonacci number $F_2 = 1$ and then split according to the numbers.

	You would first split the block in $F_3 = 2$ parts if the whole memory is to big, then procedd splitting this block it into $F_4 = 3$ parts and so on.

	If the current block at splitting level $n$ is too big, split it into $F_{n + 1}$ parts and try again.

	\item %b
	With this method you'd get more granularity for the size of memory blocks, potentially wasting less space.
\end{enumerate}

\section*{7.14}

\end{document}
